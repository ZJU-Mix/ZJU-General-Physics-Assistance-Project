\chapter[热力学]{\itr{Thermodynamics}{热力学}}
\begin{solution}[ Pressure]
    In \itr{state-of-the-art}{最先进的} \itr{vacuum systems}{真空系统}, pressures as low as $1.00\times10^{-9}\rm{Pa}$ are being attained.
    Calculate the number of molecules in a $1.00\rm{m^{3}}$ \itr{vessel}{容器} at this pressure if the temperature is 27°C.

    \tcbrule

    由理想气体方程$pV=Nk_{_B}T$得:
    \begin{equation*}
        N = \frac{pV}{k_{_B}T} = \frac{1.00\times10^{-9}\times 1.00}{1.38\times 10^{-23}\times (27+273)} = 2.40 \times 10^{11}
    \end{equation*}
    很简单的题目,一是强调热力学温度单位应统一使用开尔文;二是想说明热力学的题目套公式记得从已知条件出发。
    有的同学可能会考虑压强的微观表达公式,但这里并没有给出方均根速率,还需要结合麦克斯韦分布,这就显得繁琐了。
\end{solution}
\begin{solution}[Idea Gas]
    The root-mean-square
    speed of molecules in air (mostly $\rm{N_2}$) is comparable to
    the speed of sound in air (or in an ideal gas).\\
    (a) Using the equation of state of an ideal gas, calculate the \itr{bulk
        modulus}{体积模量} (at temperature $T$), which is defined as:
    \begin{equation*}
        B = \frac{volume \ stress}{volume \ strain} = -\frac{\Delta F/A}{\Delta V/V} = -\frac{\Delta P}{\Delta V/V}
    \end{equation*}
    (b) Recall that the speed of sound in a fluid $v=\sqrt{B/\rho}$ depends on
    the elastic and inertial properties of the fluid, where $B$ is the bulk
    modulus and $\rho$ is the density of air. Express the speed of sound
    waves in terms of molecular mass $m$, temperature $T$, as well as
    the Boltzmann's constant $k_{_B}$.\\
    (c) In fact, the speed of sound has an additional factor of $\sqrt{\gamma}$ ,
    where $\gamma$ is the \itr{adiabatic index}{绝热膨胀系数} ($\gamma = 7/5 = 1.400$ \ for \itr{diatomic molecules}{双原子分子}
    at room temperature). Compute the result in (b) at room temperature (The molar
    mass of air is $29 \rm{g/mol}$).

    \tcbrule

    读懂题目照着意思写就行。

    (a)根据$pV=nRT$,在温度确定时有$p=\dfrac{nRT}{V}$,即:
    \begin{equation*}
        \Delta p = -\frac{nRT}{V^{2}}\Delta V
    \end{equation*}
    于是代入定义即有:
    \begin{equation*}
        B = -\frac{\Delta P}{\Delta V/V} = \frac{nRT}{V}=\frac{Nk_{_B}T}{V}
    \end{equation*}
    \\
    (b)直接代入即可:
    \begin{equation*}
        v=\sqrt{\frac{B}{\rho}}=\sqrt{\dfrac{\frac{Nk_{_B}T}{V}}{\frac{Nm}{V}}}=\sqrt{\frac{k_{_B}T}{m}}
    \end{equation*}
    \\
    (c)由(b)也有
    \[v=\sqrt{\frac{k_{_B}T}{m}}=\sqrt{\frac{RT}{M}}\]
    经绝热膨胀系数的修正,即$\sqrt{\dfrac{\gamma RT}{M}}$
    代入数据,注意室温一般取$298\rm{K}$,以及把摩尔质量的单位转化为kg:
    \begin{equation*}
        v=\sqrt{\frac{\gamma RT}{M}} =\sqrt{\frac{1.400\times8.314\times 298}{29\times 10^{-3}}}\rm{m/s} = 346\rm{m/s}
    \end{equation*}
    这一结果(这里没有注意有效数字)与声速常用值$340\rm{m/s}$十分相近。
\end{solution}

\begin{solution}[The Van Der Waals Gas]
    The van der Waals equation of state is as follows:
    \begin{equation*}
        (p+\frac{a}{V^{2}})(V-b)=nRT
    \end{equation*}
    (a) Calculate the \itr{isothermal compressibility}{等温压缩系数} of the van der Waals gas in terms of $(V,T)$ and
    determine the high-temperature limit. How does this result compare to that for an ideal gas?\\
    Hint: The isothermal compressibility $\kappa_T$ is defined through $\kappa_{T}=-\dfrac{1}{V}\left(\dfrac{\partial V}{\partial p}\right)_{T}$\\
    (b) The van der Waals equation possesses a so-called critical point, where
    \begin{equation*}
        \left(\frac{\partial p}{\partial V}\right)_{T} = \left(\frac{\partial^{2} p}{\partial V^{2}}\right)_{T} = 0
    \end{equation*}
    Determine the critical pressure $p_c$, the critical volume $V_c$ and the critical temperature $T_c$. What
    is the behavior of $\kappa_T$ at the critical point?\\
    (c) Use the expressions for $V_c$, $p_c$, and $T_c$ in the van der Waals equation of state and show that it
    assumes a simple form independent of $a$ and $b$ when $T$,$V$, and $p$ are measured in terms of $T_c$,
    $V_c$, $p_c$, $i.e.$, when expressing the van der Waals equation in terms of $T/T_c$ , $V/V_c$ , $p/p_c$.

    \tcbrule

    本质仍然是一道阅读理解,理解后难度只在计算。

    (a)利用隐函数求导法则两边求导:
    \begin{equation*}
        (1-\frac{2a}{V^{3}}\frac{\partial V}{\partial p})(V-b) + (p+\frac{a}{V^{2}})\frac{\partial V}{\partial p} = 0
    \end{equation*}
    整理有:
    \begin{equation*}
        \frac{\partial V}{\partial p} = \frac{-V+b}{\dfrac{2ab}{V^{3}}-\dfrac{a}{V^{2}}+p}
    \end{equation*}
    故
    \begin{equation*}
        \kappa_T(p,V) = -\frac{1}{V}\frac{\partial V}{\partial p} = \frac{1-\dfrac{b}{V}}{\dfrac{2ab}{V^{3}}-\dfrac{a}{V^{2}}+p}
    \end{equation*}
    观察范德华状态方程,发现我们可以分离$p$:
    \begin{equation*}
        p = \frac{nRT}{V-b}-\frac{a}{V^{2}}
    \end{equation*}
    代入$\kappa_T$的表达式中,并整理得到:
    \begin{equation*}
        \kappa_T(V,T) = \frac{V^{2}(V-b)^{2}}{nRTV^{3}-2a(V-b)^{2}}
    \end{equation*}
    然后看高温极限,当$T\rightarrow +\infty$时,同样有$V\rightarrow +\infty$。为计算方便,我们取$\kappa_{T}(p,V)$,令$V\rightarrow +\infty$:
    \begin{equation*}
        \lim_{T\rightarrow +\infty}\kappa_T = \lim_{V \rightarrow +\infty}\frac{1-\dfrac{b}{V}}{\dfrac{2ab}{V^{3}}-\dfrac{a}{V^{2}}+p}=\frac{1}{p}
    \end{equation*}
    对理想气体$V = \dfrac{nRT}{p}$,其$\kappa_T$如下:
    \begin{equation*}
        \kappa_T = -\frac{1}{V}\frac{\partial V}{\partial p} = \frac{nRT}{pV^{2}} = \frac{1}{p}
    \end{equation*}
    可以看到范德华气体$\kappa_T$的高温极限与理想气体一致。
    \\

    (b)由范德华状态方程分离$p$:
    \begin{equation*}
        p = \frac{nRT}{V-b}-\frac{a}{V^{2}}
    \end{equation*}
    对$V$偏导:
    \begin{equation*}
        \left(\frac{\partial p}{\partial V}\right)_{T} = -\frac{nRT}{(V-b)^{2}} + \frac{2a}{V^{3}} = 0
    \end{equation*}
    继续对$V$偏导:
    \begin{equation*}
        \left(\frac{\partial^{2} p}{\partial V^{2}}\right)_{T} = \frac{2nRT}{(V-b)^{3}} - \frac{6a}{V^{4}} = 0
    \end{equation*}
    整理消去$nRT$即可求出$V_c$:
    \begin{equation*}
        \begin{aligned}
            \frac{2a(V-b)^{2}}{V^{3}} & = nRT = \frac{3a(V-b)^{3}}{V^{4}} \\
                                      & V_c = 3b
        \end{aligned}
    \end{equation*}
    回代得到$T_c$:
    \begin{equation*}
        T_c = \frac{3a(V_{c}-b)^{3}}{nRV_{c}^{4}} = \frac{8a}{27nRb}
    \end{equation*}
    回代到$p$的表达式中得到$p_c$:
    \begin{equation*}
        p_c = \frac{nRT_c}{V_{c}-b}-\frac{a}{V_{c}^{2}} = \frac{a}{27b^{2}}
    \end{equation*}
    回代到$\kappa_T$的表达式中,发现分母为
    \begin{equation*}
        \frac{2ab}{V_{c}^{3}}-\frac{a}{V_{c}^{2}}+p_{c} = 0
    \end{equation*}
    且分子不为$0$,故在临界点有$\kappa_T\rightarrow +\infty$。
    \\

    (c)题意其实就是用临界参数来表示$a$,$b$与$R$,这里先用量纲看一下怎么表示好,
    即用$T/T_c$,$V/V_c$,$p/p_c$代替范德华方程中的$T$,$V$,$p$,
    同时用无量纲量$a_{_0}$,$b_{_0}$,$R_{_0}$代替$a$,$b$,$R$:
    \begin{equation*}
        (\frac{p}{p_c}+\frac{a_{_0}}{\left(\frac{V}{V_c}\right)^{2}})(\frac{V}{V_c}-b_{_0})=nR_{_0}\frac{T}{T_c}
    \end{equation*}
    也即
    \[(p+\frac{a_{_0}p_{c}V_{c}^2}{V^2})(V-b_{_0}V_c) = n(\frac{R_{_0}p_{c}V_{c}}{T_{c}})T\]
    比较一下原来的形式,我们有:
    \begin{equation*}
        \begin{cases}
            a = a_{_0}p_{c}V_{c}^2 \\
            b = b_{_0}V_c          \\
            R = \dfrac{R_{_0}p_{c}V_{c}}{T_{c}}
        \end{cases}
    \end{equation*}
    接下来只需把(b)中临界参数的表达式代入,解出三个常数即可,过程略。

    记$T_R = T/T_c$,$V_R = V/V_c$,$p_{_R} = p/p_c$,则最后的形式为:
    \begin{equation*}
        (p_{_R}+\frac{3}{V_R^{2}})(V_R-\frac{1}{3})=\frac{8}{3}T_R
    \end{equation*}
\end{solution}
\begin{solution}[Heat and Work]
    An ideal \itr{diatomic}{双原子的} gas, in a \itr{cylinder}{圆柱体} with a movable \itr{piston}{活塞}, undergoes the rectangular cyclic process shown
    . Assume that the temperature is always such
    that rotational degrees of freedom are active, but vibrational modes are “frozen out.” Also assume that the only
    type of work done on the gas is \itr{quasistatic}{准静态的} compression-expansion work.
    \ctikzfig{chapter7_solution_7_4}
    (a) For each of the four steps $A$ through $D$, compute the work done on the gas, the heat added to
    the gas, and the change in the internal energy of the gas.
    Express all answers in terms of $P_1$, $P_2$, $V_1$, and $V_2$. \\
    (b) Compute the net work done on the gas, the net
    heat added to the gas, and the net change in the internal
    energy of the gas during the entire cycle. Are the results
    as you expected? Explain briefly.

    \tcbrule

    基本的$p-V$图计算,注意认清每个过程。根据题意,忽略振动对于内能的贡献,所有功均为体积功。
    双原子气体必然为线性分子,故有$i = 5$。

    (a) A为等容过程:
    \begin{equation*}
        \begin{aligned}
             & W_A = 0                                                   \\
             & Q_A = n\frac{5}{2}R\Delta T = \frac{5}{2}(p_{2}-p_{1})V_1 \\
             & \Delta U_A = Q_A = \frac{5}{2}(p_{2}-p_{1})V_1
        \end{aligned}
    \end{equation*}
    B为等压过程:
    \begin{equation*}
        \begin{aligned}
             & W_B = p_2(V_2 - V_1)                                                     \\
             & Q_B = n\left(\frac{5}{2}+1\right)R\Delta T = \frac{7}{2}p_{2}(V_2 - V_1) \\
             & \Delta U_B = Q_B - W_B = \frac{5}{2}p_{2}(V_2 - V_1)
        \end{aligned}
    \end{equation*}
    C为等容过程:
    \begin{equation*}
        \begin{aligned}
             & W_C = 0                                                   \\
             & Q_C = n\frac{5}{2}R\Delta T = \frac{5}{2}(p_{1}-p_{2})V_2 \\
             & \Delta U_C = Q_C = \frac{5}{2}(p_{1}-p_{2})V_2
        \end{aligned}
    \end{equation*}
    D为等压过程:
    \begin{equation*}
        \begin{aligned}
             & W_D = p_1(V_1 - V_2)                                                     \\
             & Q_D = n\left(\frac{5}{2}+1\right)R\Delta T = \frac{7}{2}p_{1}(V_1 - V_2) \\
             & \Delta U_D = Q_D - W_D = \frac{5}{2}p_{1}(V_1 - V_2)
        \end{aligned}
    \end{equation*}
    (b) 净功为:
    \begin{equation*}
        W = W_A+W_B +W_C+ W_D = (p_2 - p_1)(V_2 - V_1)
    \end{equation*}
    数值上等于循环对应的闭合曲线在$p-V$图中围成的面积(注意一定要是$p-V$图)

    净热量为:
    \begin{equation*}
        Q = Q_A + Q_B + Q_C + Q_D = (p_2 - p_1)(V_2 - V_1)
    \end{equation*}
    可见净热量等于净功。

    净内能变化量为:
    \begin{equation*}
        \Delta U = \Delta U_A + \Delta U_B + \Delta U_C + \Delta U_D = 0
    \end{equation*}
    内能为状态函数,由于始态与终态相同,故净内能变化量一定为0。

    对总过程有$\Delta U = Q - W$成立。这是符合预期的。
\end{solution}

\begin{solution}[Carnot Cycle]
    For a van der Waals gas, its equation of state implies a phase transition between
    liquid and gas below a critical temperature $T_c$: In the $P-V$ phase
    diagram, the \itr{isothermal line}{等温线} for a given temperature $T_{0} < T_{c}$ is not \itr{monotonically}{单调地}
    decreasing with respect to $V$, but a constant function of $V$ in some region (see the
    figure). This region corresponds to a phase transition from liquid to gas state (with a
    volume change from $V_{L}^{mol}$ to $V_{G}^{mol}$), and the mole \itr{latent heat}{潜热} is $L$ for the transition.
    Suppose we use 1 mole of this van der Waals gas/liquid mixture as the \itr{medium}{介质} for
    a Carnot cycle operating between the high temperature $T_{_0}$ and the low temperature
    $T_{_0}-\Delta T$\ —\ which are connected by two \itr{adiabatic processes}{绝热过程} $D\rightarrow A$ and $B\rightarrow C$.
    The pressure in the flat region changes from $P_{_0}$ to $P_{_0}-\Delta P$ when the temperature
    changes from $T_{_0}$ to $T_{_0}-\Delta T$.
    \ctikzfig{chapter7_solution_7_5}

    (a) Specify the heat transfer and work done in each process of $A\rightarrow B$, $B\rightarrow C$,
    $C\rightarrow D$, and $D\rightarrow A$ in such a Carnot cycle. Here, we assume that the volume
    change in $B\rightarrow C$ and $D\rightarrow A$ is negligible.\\
    (b) Calculate the total work done to the environment for this Carnot cycle and express
    its efficiency $\epsilon$ from $\epsilon = W/Q_H$. where $W$ and $Q_H$ is the total work output in
    the cycle and the heat input at the high temperature, respectively.\\
    (c) For a Carnot engine with efficiency $\epsilon = 1-\dfrac{T_C}{T_H}$, verify the \itr{Clapeyron equation}{克拉伯龙方程}:
    \begin{equation*}
        \frac{\dif{P}}{\dif{T}}=\frac{L}{T(V_{G}^{mol}-V_{L}^{mol})}
    \end{equation*}
    for the liquid/gas mixture.

    \tcbrule

    (a) 注意对理想气体成立的结论对范德华气体不再成立。

    $A\rightarrow B$是等压相变过程,由于假设,体积变化量视为$V_{G}^{mol}-V_{L}^{mol}$:
    \begin{equation*}
        \begin{aligned}
             & W_1 = p\Delta V = p_{_0}(V_{G}^{mol}-V_{L}^{mol}) \\
             & Q_1 = L
        \end{aligned}
    \end{equation*}

    $B\rightarrow C$与$D\rightarrow A$为绝热过程,故$Q_2 = Q_4 = 0$。

    又一次由题目假设,忽略$B\rightarrow C$与$D\rightarrow A$过程中的体积变化,所以认为
    \[W_2\approx 0\quad W_4\approx 0\]
    请注意,这里使用约等号而非等号是必要的。具体原因,会在本小题的最后解释。

    $C\rightarrow D$ 是等压相变过程,计算方式与$A\rightarrow B$类似:
    \begin{equation*}
        \begin{aligned}
             & W_3 = p\Delta V = -(p_{_0} - \Delta p)(V_{G}^{mol}-V_{L}^{mol}) \\
             & Q_3 = -L
        \end{aligned}
    \end{equation*}

    至此,有各过程的热与功如下:
    \[\left\{
        \begin{aligned}
            A\rightarrow B & \qquad\begin{aligned}
                                        & W_1= p\Delta V = p_{_0}(V_{G}^{mol}-V_{L}^{mol}) \\
                                        & Q_1 = L
                                   \end{aligned}
            \\[1ex]
            B\rightarrow C & \qquad\begin{aligned}
                                        & W_2\approx 0 \\
                                        & Q_2 = 0
                                   \end{aligned}
            \\[1ex]
            C\rightarrow D & \qquad\begin{aligned}
                                        & W_3 = p\Delta V = -(p_{_0} - \Delta p)(V_{G}^{mol}-V_{L}^{mol}) \\
                                        & Q_3 = -L
                                   \end{aligned}
            \\[1ex]
            D\rightarrow A & \qquad\begin{aligned}
                                        & W_4\approx 0 \\
                                        & Q_4 = 0
                                   \end{aligned}
        \end{aligned}
        \right.\]

    根据热力学第一定律(\refleaftext{law7.2}),应有
    \[\Delta U= Q_1+Q_2+Q_3+Q_4-(W_1+W_2+W_3+W_4)=0\]

    如果写成$W_2\approx 0\quad W_4\approx 0$,那么右边的等号就无法取得,这就与热力学第一定律冲突了。
    \vspace*{1ex}
    \hrule
    {
        \em
        \begin{center}
            拓展内容$^*$(不作掌握要求,补天选手自行跳过)
        \end{center}

        如果不认为$B\rightarrow C$,$D\rightarrow A$过程体积变化可忽略,那么将涉及范德华气体绝热过程下功的计算,在此介绍。

        首先,我们利用范德华方程得到$p$的表达式:
        \begin{equation*}
            p = \frac{RT}{V - b} - \frac{a}{V^2}
        \end{equation*}
        内能对$T$,$V$两个变量全微分:
        \begin{equation*}
            \dif U = \left(\frac{\partial U}{\partial T}\right)_{V} \dif T + \left(\frac{\partial U}{\partial V}\right)_{T} \dif V
        \end{equation*}
        不难注意到全微分的前一项$\left(\dfrac{\partial U}{\partial T}\right)_{V} = nC_{V}$,至于后一项,我们需要先得到普适的能态方程。

        熵对$T$,$V$两个变量全微分:
        \begin{equation*}
            \dif S = \left(\frac{\partial S}{\partial T}\right)_{V} \dif T + \left(\frac{\partial S}{\partial V}\right)_{T} \dif V
        \end{equation*}
        代入热力学基本方程中:
        \begin{equation*}
            \dif U = T\dif S - p \dif V
            = T\left(\frac{\partial S}{\partial T}\right)_{V} \dif T + \left[T\left(\frac{\partial S}{\partial V}\right)_{T} - p\right] \dif V
        \end{equation*}
        定义亥姆霍兹自由能$A=U-TS$,关于其热力学基本方程为
        \begin{equation*}
            \dif{A} = \dif U - \dif (TS) = \dif U - T \dif S - S \dif T = -S\dif{T} - p\dif{V}
        \end{equation*}
        利用二元函数全微分的必要条件,即偏导次序可交换,有下面的关系,称为麦克斯韦关系(之一):
        \begin{equation*}
            \left(\frac{\partial S}{\partial V}\right)_{T}
            =\left[\frac{\partial}{\partial V}\left(\frac{\partial A}{\partial T}\right)_V \right]_T
            =\left[\frac{\partial}{\partial T}\left(\frac{\partial A}{\partial V}\right)_T \right]_V
            =\left(\frac{\partial p}{\partial T}\right)_{V}
        \end{equation*}
        从而得到下面的式子,事实上被称为能态方程:
        \begin{equation*}
            \left(\frac{\partial U}{\partial V}\right)_{T} = T\left(\frac{\partial p}{\partial T}\right)_{V} - p
        \end{equation*}
        将范德华气体的状态方程代入,有:
        \begin{equation*}
            \left(\frac{\partial U}{\partial V}\right)_{T} = \frac{RT}{V-b} - p = \frac{a}{V^{2}}
        \end{equation*}
        这样我们就得到了范德华气体内能的微分形式:
        \begin{equation*}
            \dif U = nC_{V}\dif T + \frac{a}{V^{2}}\dif V
        \end{equation*}
        积分有(可以证明,事实上范德华气体的$C_{V} = C_{V}(T)$,与体积无关):
        \begin{equation*}
            \Delta U = \int_{T_1}^{T_2}C_{V}\dif T + \frac{a}{V_1} - \frac{a}{V_2}
        \end{equation*}
        从而根据热力学第一定律$\Delta U = -W$:
        \begin{equation*}
            \begin{aligned}
                 & W_2 = \int_{T_{_0}-\Delta T}^{T_{_0}}nC_{V}\dif T - \frac{a}{V_{B}} + \frac{a}{V_{G}^{mol}}  \\
                 & W_4 = -\int_{T_{_0}-\Delta T}^{T_{_0}}nC_{V}\dif T + \frac{a}{V_{A}} - \frac{a}{V_{L}^{mol}}
            \end{aligned}
        \end{equation*}

    }
    \vspace*{1ex}
    \hrule
    \vspace*{1ex}
    (b) 利用(a)中的结果代入即可:
    \begin{equation*}
        \epsilon = \frac{W}{Q_H} =\frac{W_1}{Q_1}=\frac{\Delta P(V_{G}^{mol}-V_{L}^{mol})}{L}
    \end{equation*}

    (c) 在这里$T_H = T_{0}$,$T_C =T_{0} - \Delta T$,代入有:
    \begin{equation}
        \epsilon = \frac{T_H-T_C}{T_H}=\frac{\Delta T}{T_{0}}
    \end{equation}

    结合(b)知
    \begin{equation}
        \epsilon = \frac{\Delta P(V_{G}^{mol}-V_{L}^{mol})}{L}
    \end{equation}

    联立式(7.1),(7.2),知
    \[
        \frac{\Delta T}{T_{0}}=\frac{\Delta P(V_{G}^{mol}-V_{L}^{mol})}{L}
    \]
    对近平衡可逆过程,用$\dif T$与$\dif P$代替$\Delta T$与$\Delta P$,并移项,即有
    \begin{equation*}
        \frac{\dif{P}}{\dif{T}}=\frac{L}{T(V_{G}^{mol}-V_{L}^{mol})}
    \end{equation*}
\end{solution}
\begin{solution}[Entropy]
    Consider the Carnot cycle operating with a hot and cold heat baths whose temperatures are $T_h$ and
    $T_c$ ($T_h > T_c$), respectively. Let working \itr{substance}{材料} be the gas in the engine, and we consider a gas in
    general for the working substance.\\
    (a) Suppose that the amount of heat exchange during the \itr{isothermal process}{等温过程} with the hot (cold) heat bath is
    $Q_h$ ($Q_c$), determine the entropy change $\Delta S_h$ and $\Delta S_c$ of the working substance in the respective
    process. Then, specify $\Delta S_h$ and $\Delta S_c$ are positive or negative. Here, take the positive sign of $Q_h$ and $Q_c$ for the heat input from the heat bath to the \linebreak working substance.\\
    (b) Now we use an ideal gas as the working substance and consider a free expansion process. Suppose the
    initial temperature of the gas is $T_i$ and the volume of the gas increases from $V_i$ to $V_f$. Determine the
    heat input $Q_{fe}$ and the entropy change $\Delta S_{fe}$ of the working substance through the free expansion
    process.\\
    (c) By replacing a \itr{quasi-static}{准静态} isothermal expansion process in the Carnot cycle by a free expansion
    process, it seems that it is possible to construct a cycle with a single heat bath. Does this fact
    \itr{violate}{违背} the second law of thermodynamics? Answer by yes or no, then explain your answer using
    the case of the Carnot cycle.

    \tcbrule

    (a) 等温过程,高温时系统吸热,低温时系统放热,代入熵的定义中:
    \begin{equation*}
        \begin{aligned}
             & \Delta S_h = \frac{Q_h}{T_h} \\
             & \Delta S_c = \frac{Q_c}{T_c}
        \end{aligned}
    \end{equation*}
    此处$Q_h$为正,$Q_c$为负,故$\Delta S_h$为正,$\Delta S_c$为负。

    (b) 根据自由膨胀,有条件外界真空($W_{fe} = 0$)与绝热($Q_{fe} = 0$)。

    熵变则利用熵是状态函数与自由膨胀始态与终态温度相等的特性,用等温过程\mgnote{见\refleaftext{chapter7_process_introduction}}连接始态与终态,熵变为:
    \begin{equation*}
        \Delta S_{fe} = nR\ln{\frac{V_f}{V_i}}
    \end{equation*}

    (c) 否。因为自由膨胀并不是一个可逆过程,压缩时会导致外界环境发生变化,符合热力学第二定律。

    若从利用公式的角度解释,则由热力学第一定律,对一个循环有
    \[\Delta U=Q-W=Q_{fe}+Q_c-W=Q_c-W=0\]

    由$Q_c<0$知$W<0$,即外界需要对系统做功,这符合热力学第二定律。
\end{solution}
