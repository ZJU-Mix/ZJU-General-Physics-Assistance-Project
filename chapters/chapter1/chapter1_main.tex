	\chapter[测量]{\itr{Measurement}{测量}\mgnote*{\raggedleft 作者:23级-陈锦浩}}
	本章节内容较为简单,一般相关题目会出现在日常上课的课后小测与作业内,大型考试基本不涉及。重点是掌握物理学的基本单位,以及利用国际单位制以及相关比例关系推导特殊物理量的单位表达式与可能的公式表达式。
	\section[单位与量纲]{\itr{Unit \& Dimension}{单位与量纲}}
	\begin{Itemize}
		\item \itr{Basic Quantity}{基本量}\ \eg Length($L$), Mass($M$), Time($T$)\\
		在自然世界中,我们会用各种物理量来描述物质的性质。就像在平面几何中我们可以基于几大公理推出各种定理,在物理中,我们也可以规定一些“基本的量”,使得所有的物理量可以由这些量导出。自然地,我们就称呼这些量为基本量。
		\item \itr{Unit}{单位}在规定好一些物理量之后,我们会想办法去度量它们。既出于定量测量的要求,又出于减少物理公式参数的考量,人们定义了各种各样的单位。
		\begin{itemize}
			\item \itr{SI Units}{国际单位制单位}\mgnote{SI $\Leftrightarrow$ International System of Units}\ \eg meter(m), kilogram(kg), second(s)
			\item \itr{Non SI Units}{非国际单位制单位}\ \eg mm, $\upmu$m, g\\
			\En{If you \itr{convert}{转化} everything to the basic SI units, you can \itr{omit}{省略} them during the calculation, but remember to put back the correct units at the end.}\\
			事实上,可以在计算过程中省略的约定也算是国际单位制存在的意义之一(\dove :大概吧)。
		\end{itemize}
		\item \itr{Dimension}{量纲}\ 一个物理量的量纲指的是这个物理量关于基本量的导出式。
		
		\eg 速度的量纲可以表示为[$v$]=$\dfrac{L}{T}$
		
		\item \itr{Dimension Analysis}{量纲分析}\ 量纲分析是检验物理公式正确性的好方法。简而言之,你可以简单地通过分析一个物理等式左右两边的量纲是否一致来初步判断这个等式可不可能成立。当然,量纲分析也可以作为导出实验公式大方向的一个指导,又或者……是你忘记了公式中的某一个物理量时尝试硬添物理量时的救命稻草。
	\end{Itemize}
	\section[数据处理]{\itr{Data Processing}{数据处理}}
	这部分普通物理学实验\Romannumeral{1}\footnote{说到这里,\dove 还编写了一个用于普物实验数据处理的软件,链接为 \url{https://github.com/CrazySpottedDove/Lab-Assistance.git},如有需要,可以自行下载。}会详细说明并实际计算运用,非常抱歉,此处略去。