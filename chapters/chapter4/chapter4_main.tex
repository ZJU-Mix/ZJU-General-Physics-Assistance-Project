\begin{comment}
    \documentclass{Physics_H_Notes}
    \usepackage{upgreek}
    \begin{document}
    \end{comment}
        \chapter[流体力学]{\itr{Fluid Mechanics}{流体力学}}
        本章节存在一定的特殊性:在2022年及之前,流体力学是普物\Romannumeral{1}的内容,而在2023级,这一块内容被路欣老师为代表的教学组删去了。

        当看到流体力学的时候,你想到的是什么?或许是初中的水的压强,又或者是湍流等复杂情景。其实都无所谓,毕竟它们都是流体的研究。当今世界对于流体力学的探究也是一个重要的力学方向,小到潜水,大到飞机飞船,都需要关注流体对于物体的影响。
        但是,我们这门课程的流体力学相当简单和浅层,按照笔者当时授课较为简单的pzq老师的内容,我用一句话定义这一块内容:两个公式走天下!这是由于我们使用了大量的理想情景,抛除复杂情景所致。
        \section[流体的定义与性质]{\itr{Definition \& Properties of Fluids}{流体的定义与性质}}
        生活中的物体可分为固体、液体和气体三大类,其中,我们一般将液体和气体统称为流体。它们一般受到的力是不一样的:
        \begin{Itemize}
            \item \itr{solid}{固体}:\itr{compression force}{压缩力}(指物体能够从两侧被挤压), \itr{tensile force}{拉力}(指物体能够受力被拉伸), 
                \itr{shearing force}{剪切力}(指物体能够受到不在同一直线上的两个相反方向的力而保持形状不变)
            \item \itr{liquid}{液体}:compression force, tensile force, no shearing force
            \item \itr{gas}{气体}:compression force, no tensile force, no shearing force
        \end{Itemize}
        
        在静止的液体中,我们已经了解到物体一定会受到水的压力,进而定义压强。
        \begin{law}[静止流体中的压强]
            利用微元法分析,我们可以知道物体受到的静止流体的压强为:
            \begin{center}
                $\dfrac{dp}{dy}=-\rho g$,积分可得$p=p_0+\rho gh$
            \end{center}

            其中$p_0$是液体表面的压力,$h$为深度,$y$为该点相对于参考平面的高度。
        \end{law}
        
        对于静止流体,我们也有两个重要的而且很直观的性质:
        \begin{law}[\itr{Pascal's Principle}{帕斯卡原理}]
            施加在封闭流体上的力被不减弱地传递到流体的每一部分和容器的壁上。假设流体中受力面积为$S$的某点处原有压强为$p_0$,在其上方平面施加垂直向下的力F,则该点处当前的压力为:
            \begin{center}
                $F^{'}=p_0 S +F$
            \end{center}
        \end{law}

        这是很容易理解的,我们以按压海绵为例,力的作用会逐渐传导到物体的每个位置。当然,帕斯卡原理强调了在流体中这个力是不会减弱的,
        而海绵是固体,力是会减弱的。
        \begin{law}[\itr{Archimedes's Principle}{阿基米德原理}]
            一个部分或者完全浸入液体中的物体受到的浮力等同于它所排出的液体的重力。
        \end{law}

        最简单的例子,就是我们初中学过的物体受到水的浮力的公式:
        \begin{center}
            $F=\rho gV$
        \end{center}

        我们可以很容易地知道这个公式的本质就是阿基米德原理。

        在液体中存在表面张力,表面张力是一种由于液体分子间的相互吸引与拉扯而产生的力的作用,其大小与物体间的接触长度有关。定义表面张力系数为:
        \begin{center}
            $\gamma =\dfrac{F}{l}$
        \end{center}

        其中$F$为该处受到的表面张力大小,$l$表示接触面的长度。这里的表面张力定义与普通化学(H)中的定义相同。当表面张力产生时,由于物体具有一定厚度,
        一般情况下会产生两个液体膜的拉力作用,因此计算时通常需要将长度以两倍长度,即$2l$计算。

        由于物体与液体间的表面张力系数不同,会产生浸润与非浸润的区别。这个概念与普化课程一致,且一般不作考察,因此我们暂不讨论。
        \section[流体动力学]{\itr{Fluid Dynamics}{流体动力学}}
        或许你一直有疑问:我感觉我也没学啥呀,怎么就动力学了?确实,我们到目前的流体力学都是初中水平。而接下来,就是流体力学两个方程走天下的经典例子。
        
        我们定义流体的通量为:$Q=\dfrac{\delta m}{\delta t}=\rho Av$,表示单位时间内流过截面积为$A$的流体的质量,其中$\rho$表示流体密度,$A$表示流体通过的截面面积,$v$表示通过这个截面时的流体流速。
        首先考虑一个水管,里面充满了水,水管不可压缩变形,水也不可压缩变形。那么很明显,水管的一端进入多少水,另一端就会有多少水流出。
        \begin{law}[管流原理(连续性原理)]
            流体流入某个管的通量等于流出该管的通量,即:$\rho_1 A_1 v_1=\rho_2 A_2 v_2$
        \end{law}

        在这个方程中,我们看到了同一个流体的管流特点。当然,这个方程忽略了压强和位置的变化。因此,我们针对理想流体,即不可被压缩、遵循管流原理、没有湍流现象或其他扰动的流体,提出了伯努利方程:
        \begin{law}[\itr{Bernoulli’s Equation}{伯努利方程}]
            \centering
            $\dfrac{1}{2}\rho v^2 +p+\rho gy=constant$
        \end{law}

        只需要在使用中注意的一点是,这里的$y$表示的是位置相对于参考平面的高度,而非相对于液面的深度。在满足上述条件的情况下,只要是同一个连通的流体,在其内部每一点处,上述等式左边的结果均相等。

        当然,看到这里,我们已经了解了两个重要的方程。结合中学的压强的定义,我们已经可以用两个方程大步走入流体力学的世界了。最后,这些已经足够我们普物的内容了,更为复杂的情况,交给流体力学的专家们解决吧
        %\end{document}