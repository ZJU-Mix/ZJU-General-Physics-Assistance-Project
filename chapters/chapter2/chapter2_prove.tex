\chapter[质点运动学与动力学]{\itr{Fundamentals of particle kinematics and dynamics}{质点运动学与动力学基础}}
\begin{prove}[Newton's First Law of Motion]
    在十七世纪以前,人们普遍认为力是维持物体运动的原因。用力推车,车子才前进,停止用力,车子就要停下来。古希腊的哲学家亚里士多德(公元前384~322)根据这类经验事实得出结论说:必须有力作用在物体上,物体才能运动,没有力的作用,物体就要静止下来。

    在亚里士多德以后的两千年内,动力学一直没有多大进展。直到十七世纪,意大利的著名物理学家伽利略才根据实验揭示了现象的本质,指出了亚里士多德的观点的错误。伽利略发现,运动物体之所以会停下来,是因为受到摩擦阻力的缘故。他断言:一旦物体具有某一速度,只要没有加速或减速的原因,这个速度将保持不变,而这种情况只有在摩擦力极小的水平面上才能近似达到。根据这种观点看来,力不是维持物体的运动即维持物体的速度的原因,而是改变物体运动状态即改变物体速度的原因。
    
    伽利略是怎样得到这个结论的呢?伽利略并没有脱离日常经验,而是对经验进行了分析。他研究了物体在斜面上的运动,发现物体沿斜面向下运动,有加速的原因出现,速度不断增加;沿斜面向上运动,有减速的原因出现,速度不断减小。他根据这一事实进行推论,指出在没有倾斜的光滑水平面上,物体的运动应当是既没有加速也没有减速,速度应当是不变的。当然,伽利略知道,由于物体受到摩擦力的阻碍,这种水平运动的速度实际上并不是不变的。摩擦越小,物体以接近于恒定速度运动的时间就越长。在没有摩擦的理想情况下,物体将以恒定的速度持续运动下去。
    
    我们可以用现代的实验设备来近似地验证上述结论。把物体放在一个水平导轨上,并设法使物体和导轨之间形成气层,物体沿这种气垫导轨运动时摩擦很小。推动一下物体,可以看到物体沿气垫导轨的运动很接近匀速直线运动。
    
    伽利略还根据下面的理想实验进行推论,如图2.1:$A$所示,让小球沿一个斜面从静止滚下来,小球将滚上另一个斜面。如果没有摩擦,小球将上升到原来的高度,他推论说,如果减小第二个斜面的倾角(图2.1:$B$),小球在这个斜面上达来的高度,而要沿着水平面以恒定速度持续运动下去,到原来的高度就要通过更长的距离。继续减小第二个斜面的倾角,使它最终成为水平面(图2.1:$C$),小球就再也达不到原来的高度,而是要沿着水平面以恒定速度持续运动下去。
    
	\ctikzfig{chapter2_newton's first law}
	\begin{center}
		图 2.1:伽利略的理想实验
	\end{center}
    伽利略同时代的法国科学家笛卡儿(1596~1650)进一步补充和完善了伽利略的论点,第一次明确地表述了惯性定律。笛卡儿认为:如果没有其他原因,运动的物体将继续以同一速度沿着一条直线运动,既不会停下来,也不会偏离原来的方向。这样,笛卡儿为发展动力学又迈出了重要一步。
    
    牛顿在伽利略等人的研究基础上,并根据他自己的研究,系统地总结了力学的知识,提出了牛顿第一定律:一切物体总保持匀速直线运动状态或静止状态,直到有外力迫使它改变这种状态为止。
    
    以上内容引自 \url{https://enjoyphysics.cn/Article1873}。
\end{prove}

\begin{prove}[Newton's Second Law of Motion]
    牛顿第二定律主要来自伽利略的自由落体定律。
    
    伽利略已经通过实验证明自由落体运动是匀加速运动,即加速度不变的运动。只要物体在自由下落时它的重量(所受到的重力)保持不变,就说明物体在匀加速运动中受到的作用力是恒定不变的。如果说,一个物体在地面上的重量跟它在距地面$10$英尺高处的重量一样或基本一样,质量也保持不变,这在当时是不会有任何争议的。所以,牛顿第二定律是从自由落体定律延伸出来的。
    
    另外,托里拆利已经运用过一个等式:$\vec{F}t=\Delta (m\vec{v})$,即恒力与作用时间的乘积和动量的变化相等。这就是我们现在熟知的动量定理。如果这个等式两边同时除以时间$t$,那么就可以得出$\displaystyle \vec{F}=m\frac{\Delta \vec{v}}{t}$。$\displaystyle\frac{\Delta \vec{v}}{t}$就是速度变化率,即加速度。
\end{prove}

\begin{prove}[Newton's Third Law of Motion]
    十七世纪中叶,碰撞问题成为科学界共同关心的课题,不少科学家都致力于该问题的研究。当时,对碰撞问题研究较早的有笛卡尔。1664年,牛顿受到笛卡尔的影响,也开始研究二个球形非弹性刚体的碰撞问题。1665—1666年间,牛顿又研究了两个球形刚体的碰撞问题。他没有像其它科学家那样把注意力集中在动量和动量守恒方面,而是把注意力放在物体之间的相互作用上,对于两刚体的碰撞,他提出,“……一于是在它们向彼此运动的时间中(就是它们相碰的瞬间)它们的压力处于最大值,……它们的整个运动是被此一瞬间彼此之间的压力所阻止,……只要这两个物体都不互相屈服,它们之间将会持有同样猛烈的压力,……它们将会象以前弹回之前彼此趋近那样多的运动相互离开。”

    上面这段话可看出,牛顿当时就已认识到在物体相互碰撞的瞬间,它们的运动被彼此之间的压力所改变。稍后,牛顿又认识到:“如果二物体$p$和$r$彼此相遇,因为$p$压$r$和$r$压$p$是一样大小,所以二者的阻力是相同的。”同时,他还用图形明确表示$p$压$r$和$r$压$p$的力是在同一条直线上。
\end{prove}
\begin{prove}[Newton's Second Law in non-inertial frame]
	在惯性参考系中,牛顿第二定律$\vec{F}=m\vec{a}$成立,其中$\vec{F}$与$m$不会随参考系的变化而变化。这时,看另一个非惯性参考系,由于它相对于惯性参考系不保持匀速直线运动或静止,这个参考系中同一物体的加速度$\vec{a}'$显然不同于原惯性参考系中的$\vec{a}$。如此,便有$\vec{F}=m\vec{a}\ne m\vec{a}'$,即牛顿第二定律不成立。
	
	那么,在这个非惯性参考系中,如何去修正牛顿第二定律呢?假设非惯性参考系相对惯性参考系的加速度为$\Delta \vec{a}$,则有$\vec{a}=\vec{a}'+\Delta \vec{a}$,那么$\vec{F}=m\vec{a}=m\vec{a}'+m\Delta \vec{a}$,也就是$\vec{F}-m\Delta \vec{a}=m\vec{a}'$。这可以认为是在这个非惯性参考系中全新的牛顿第二定律。然而,若是这样,每个非惯性参考系都有不同的牛顿第二定律,这显然是不方便、不合理的。
	
	虽然修正牛顿第二定律是不合理的,但是我们可以去修正力。引入虚拟力$\vec{f}_{fictitious}=-m\Delta \vec{a}$,那么在上面的非惯性参考系中,就有
	\[\vec{F}+\vec{f}_{fictitious}=m\vec{a}'\]
	我们将真实力与虚拟力的合力称为表现力,即
	\[\vec{F}_{effective}=\vec{F}+\vec{f}_{fictitious}\]
	
	如此,对于非惯性参考系的表现力来说,牛顿第二定律依然成立,这样就极大的方便了我们的计算。
\end{prove}
\begin{prove}[Work-kinetic energy theorem]
\begin{equation}
    \begin{aligned}
 \sum W&=\int_{x_{i}}^{x_{f}}(\sum F_{x})\dif x=\int_{x_{i}}^{x_{f}} ma_{x}\dif x\\[1ex]
a_x&=\frac{\dif v}{\dif t}=\frac{\dif v}{\dif x}\frac{\dif x}{\dif t}=v\frac{\dif v}{\dif x}\\[1ex]
\Rightarrow\sum W&=\int_{x_{i}}^{x_{f}}mv\frac{\dif v}{\dif x}\dif x=\int_{v_{i}}^{v_{f}}mv\dif v=\frac{1}{2}mv_{f}^{2}-\frac{1}{2}mv_{i}^{2}
    \end{aligned}
    \nonumber
\end{equation}
大家有没有疑惑,动能的公式和动能定理,到底谁是鸡谁是蛋?网上的一些动能定理的推导,都是基于$\frac{1}{2}mv^{2} $就是物体动能。然而网上物体动能的推导,却是基于外力做功等于物体动能增加量的动能定理。这样的循环论证很让人疑惑。个人认为,动能定理属于能量守恒定律的范畴,因为功是能量转化的量度,是能量守恒定律中的桥梁。在此基础上,人为规定了合理的动能的表达式,而且经过实验和理论的验证,这个定义是合理,体系是自洽的。至于能量守恒定律为什么是对的...基于实验?基于经验?详见热力学吧。
\end{prove}

\begin{prove}[Impulse-Momentum Theorem]

    由牛顿第二定理就可以推出:
    \begin{equation}
        \begin{aligned}
     &\vec{F}=m\frac{\mathrm{d}\vec{v}}{\mathrm{d}t}=\frac{\mathrm{d}(m\vec{v})}{\mathrm{d}t}=\frac{\mathrm{d}\vec{p}}{\mathrm{d}t}\\
    \Longrightarrow &\vec{F}\mathrm{d}t=\mathrm{d}\vec{p}\\
    \Longrightarrow &\vec{I}=\int_{t_{i}}^{t_{f}}\vec{F}\mathrm{d}t= \int_{p_{i}}^{p_{f}}\mathrm{d}\vec{p}=m\vec{v_{f}}-m\vec{v_{i}} 
        \end{aligned}
        \nonumber
    \end{equation}
\end{prove}
\begin{prove}[质心系中,系统的总动量为$\vec{0}$]
	设有 \( N \) 个质点,质量分别为 \( m_1, m_2, \ldots, m_N \),位矢为 \( \vec{r}_1, \vec{r}_2, \ldots, \vec{r}_N \)。质心的位置 \(\vec{R}\) 定义为:
	
	\[
	\vec{R} = \frac{1}{M} \sum_{i=1}^{N} m_i \vec{r}_i
	\]
	
	其中 \( \displaystyle M = \sum_{i=1}^{N} m_i \) 是系统的总质量。
	
	在质心系中,质点的位矢变为:
	\begin{equation}
		\vec{r}_i' = \vec{r}_i - \vec{R}
	\end{equation}
	
	任意参考系中,系统的总动量 \(\vec{P}\) 为:
	
	\[
	\vec{P} = \sum_{i=1}^{N} \vec{p}_i = \sum_{i=1}^{N} m_i \vec{v}_i
	\]
	
	其中 \(\vec{v}_i\) 是对应质点的速度矢量。代入式(2.1)可得:
	
	\[
	\vec{P} = \sum_{i=1}^{N} m_i \left(\frac{\dif\vec{r}_i'}{\dif t} + \frac{\dif\vec{R}}{\dif t}\right)
	\]
	
	由质心的定义,质心速度 \(\vec{V}\) 为:
	
	\[
	\vec{V} = \frac{\dif \vec{R}}{\dif t} = \frac{1}{M} \sum_{i=1}^{N} m_i \vec{v}_i
	\]
	
	因此,
	
	\[
	\vec{P} = \sum_{i=1}^{N} m_i \frac{d\vec{r}_i'}{dt} + M \vec{V}
	\]
	
	注意到在质心系中,\(\vec{V} = \vec{0}\),因此
	
	\[
	\vec{P}^{CM} = \sum_{i=1}^{N} m_i \frac{\dif\vec{r}_i'}{\dif t}
	\]
	
	又由质心系的定义,有
	
	\[
	\sum_{i=1}^{N} m_i \vec{r}_i' = \vec{0}
	\]
	
	综上可知,质心系中系统总动量为零,即
	
	\[
	\vec{P}^{CM} = \vec{0}
	\]
\end{prove}
\begin{prove}[Newton's Law of Universal Gravitation]
一方面,平方反比的规律是牛顿根据开普勒的行星周期与它们的距离轨道中心的距离的二分之三次方成正比的规律(开普勒行星运动第三定律)得出的结论。牛顿从一个完完全全的运动学结论中得出了动力学的结论:

由开普勒第三定律:
\[
\frac{r^{3}}{T^{2}}=K
\]

由圆周运动的规律:
\[
F=m\frac{v^{2}}{r}
\]

加之
\[
v=\frac{2\pi r}{T}
\]

就有
\[F=\frac{4\pi^{2}mK}{r^{2}}\]

另一方面,引力同时与两物体质量成正比的规律,是基于牛顿自己的三个定律。根据第三定律,力的作用是相互的,行星受到的引力$F$与其质量$m$成正比,同理,太阳受到的力$F'$与太阳的质量$M$成正比,而$F=F'$,则$F$必同时与$m$和$M$成正比。
\end{prove}